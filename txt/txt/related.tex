\section{Related Work}
\subsection{3D Druck und Additive Fertigung von Gebäuden}
\subsection{Legeroboter}
\subsection{Materialien}
\subsection{Bausteine}
\subsection{Mauer detailing und das (3D) Bin Packing Problem}
TODO hinführen über Bin Packing hin zu "spezialfall" Wall detailing mit arbiträren Bausteinen und Eigenschaften (wie versetzen der ziegel)

TODO über bin packing schreiben, erklären paper suchen, lösungsansätze zu np hartem problem 

Xu Chengran et al. haben in ihrem Paper "Optimal brick layout of masonry walls based on intelligent evolutionary algorithm and building information modeling" verschiedene Optimierungsansätze aus dem Bereich des 2D Packaging Problems getestet \cite{Xu2021}.
%TODO was ist das für ein Problem? Zitat aus nem Paper finden!
Konkret wurden drei Algorithmen verwendet: Differential Evolution, Particle Swarm Optimization und Neighbourhood Field Optimization.
%TODO ergebnisse vergleichen und eines davon hervorheben, welches ich evtl selbst einbau
Außerdem wird ein drei-phasiges Vorgehen vorgeschlagen: Data collection, Brick layout und Data Output.
%TODO das vmtl einfach auch so aufziehen. Mauern aus modell extrahieren mit geometrischen infos, optimieren und iwie rausballern
Dieses Vorgehen eignet sich auch für das Finden von Bausteinkonfigurationen in dieser Arbeit, da zuerst alle relevanten geometrischen Daten (in diesem Fall Wände, Fenster, Türen usw.) aus dem 3D Modell gesammelt werden müssen, bevor das Detailing stattfinden kann.
Nach dem Optimieren der Bausteinkonfiguration muss das Ergebnis ebenfalls in ein Format gebracht werden, das für die folgenden Schritte verwendet und eventuell auch dem Nutzer angezeigt werden kann.


Soft items: https://arxiv.org/abs/2206.15116

Irregular Shaped items: https://link.springer.com/content/pdf/10.1631/FITEE.1400421.pdf

"Parametric Blockwall-Assembly Algorithms for the Automated Generation of Virtual Wall Mockups Using BIM"
