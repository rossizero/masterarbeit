\chapter{Aufgabenstellung}
Nachfolgend werden in chronologischer Reihenfolge alle notwendigen Teilaufgaben aufgezeigt, die gelöst werden müssen, um ein Gebäude zu planen, welches anschließend von Robotern gebaut werden kann.

\section{Modellierung des 3D Modells}
Es muss ein Weg gefunden werden, dem Nutzer das Modellieren von Teilstücken eines Gebäudes zu erleichtern, sodass er keine einzelnen Bauteile in dem Editor verwenden muss, sondern Konglomerate (wie etwa eine Wand oder ein Fenster) zur Verfügung hat, welche als Ganzes bewegt oder verformt werden können.
Dabei dürfen aber Informationen über die eigentlich zugrundeliegenden Bauteile (z.B. Ziegel) entweder nicht verloren gehen und implizit vorgegeben sein (z.B. durch das aufzwingen eines Rasters, welches dem Formfaktor der Bauteile entspricht, sodass etwa eine Wand nicht in beliebig kleinen Schritten vergrößert werden kann, sondern immer nur einen Längensprung um die Länge des kleinsten zugrundeliegenden Ziegel macht) oder das Programm versucht, das Gebäude nach beenden des Designprozesses möglichst gut mithilfe des vorgegebenen Sets an Bauteilen abzubilden. Das führt aber dazu einen möglichst guten Kompromiss finden zu müssen, was durch ein Raster von vornherein vermieden werden kann.

Fragestellung: Wie kann das Modellieren eines Gebäudes für Nutzer vereinfacht werden ohne wichtige Informationen zu verlieren?

\section{Finden eines geeigneten Dateiformats}
Wie breits vorweg genommen, muss es möglich sein aus dem Gebäudemodell eine Menge an Bauteilen von vorgegebenem Typen zu berechnen, die das Gebäude komplett (oder möglichst gut) abbildet.
Das geht entweder, indem man diese Menge während dem Designprozess impliziert vorliegen hat, da alle größeren Teilstücke (wie etwa eine Wand) schon mithilfe der zugrunde liegenden Bauteile definiert wurden oder durch das einmalige Konvertieren eines von den Bauteilen völlig losgelösten 3D Modells in eben jene Menge an Bauteilen.
Im ersten Fall fällt das Konvertieren weg und der Nutzer arbeitet immer direkt auf dem Datenformat, welches anschließend in späteren Schritten verwendet wird.
Im zweiten Fall kann das konvertierte Modell vermutlich nicht wieder in den Editor zurückgeladen werden, sodass man ein "Design Speicherformat" und ein "Konvertiertes Datenformat" hätte.
Dies ist aber ebenfalls eine gängige Praxis bei z.B. Vektorgraphik-Bearbeitungsprogrammen, die in einem proprietären Format Daten über mathematisch definierte Kurven halten und der Nutzer daraus Bilder in pixelbasierte Formaten wie jpg oder png generieren kann.
Dabei gehen viele Informationen verloren und man kann das exportierte Bild nicht in derselben Art wieder in das Programm laden.
Mit Blick auf die Domäne des Häuserbauens ist allerdings ein möglichst großer Freiheitsgrad wünschenswert, was für den Konvertierungsansatz spricht.
Dabei wird es wie bereits erwähnt herausfordernd eine möglichst gute Abbildung des im Editor designten Gebäudes mithilfe eines limitierten Sets an Bauteiltypen zu finden, welches nach Möglichkeit alle vom Nutzer gewünschten Eigenschaften behält.
Als Speicherformat kann ein im Bauingeneursbereich weit etablierter Standard verwendet werden, der neben den bloßen geometrischen Informationen über das Gebäude auch wichtige Details aus anderen Fachbereichen integriert.
Dieser Standard wird im weiteren Verlauf dieses Expos\'{e}s vorgestellt.

Fragestellung: Das Gebäude im Hintergrund immer als Menge von definierten Bauteilen oder als gängiges 3D Modell speichern, welches irgendwann in Bauteile überführt werden muss?

\section{Finden eines geeigneten Bauplans für Roboterschwärme}
Um dem Umfang und die Komplexität diesen Schrittes zu erfassen, wurde Ludwigs Dissertation gelesen \cite{Naegele2021}.
Diese beschäftigt sich mit dem Finden geeigneter Baupläne für beliebige Produkte mit Einbeziehung diverser (etwa geometrischer) Einschränkungen der Einzelteile oder der Monatageroboter.
Mit einem solchen Verfahren kann die derzeit teure Individualisierung von Produkten kosteneffizient ermöglicht werden, da sämtliche Produktionsschritte automatisch herausgefunden werden und damit Zeit sowohl in der Planung, als auch beim Einlernen der Montageroboter gespart werden kann.
Im Grunde ist die Problemstellung aus dieser Arbeit folgende:
Die Menge möglicher Baupläne für ein Produkt steigt exponentiell mit der Menge der verwendeten Bauteile und kann als sehr großer Suchraum für passende Lösungen angesehen werden, welcher z.B. mithilfe bestimmter Anforderungen an den resultierenden Bauplan verkleinert werden kann.
Tatsächlich kann diese Fragestellung direkt auf die Domäne des Häuserbauens angewendet werden:
Bei Gebäuden handelt es sich um Objekte mit vielen Tausend Bauteilen, was aufgrund des exponentiellen Wachstums des Suchraums zu einer unmöglich großen Menge potentieller Baupläne führt.
Um dem entgegenzuwirken, sucht man Constraints, wie etwa die Erkenntnis, dass Ziegel nur von unten nach oben aufgeschichtet werden können.
Damit fallen eine Vielzahl an Bauplänen weg und der Suchraum verkleinert sich erheblich.
Auch die Beschaffenheit der Bauteile und der Roboterendeffektoren und -körper bringen Constraints mit sich.
Allerdings ist der Einsatz von mehreren Robotern ein neuer Aspekt, der in Ludwigs Arbeit nicht berücksichtigt wird.
Dennoch liegt es nahe die Ergebnisse aus seiner Arbeit für dieses Projekt zu verwenden und für die Abarbeitung durch mehrere Agenten anzupassen.
Als Input für seinen "Suchalgorithums" wurde in Ludwigs Dissertation ein eigenes xml-artiges Datenformat zur vollständigen Beschreibung des fertigen Bauteils eingeführt auf Basis dessen ein nach einer Heuristik optimierter Bauplan herausgesucht wird.
Oftmals hat diese Heuristik zum Ziel, die Montagedauer oder die Kosten der Monatge zu minimieren.
Es ergibt Sinn dieses Format als Ergebnis aus dem vorherigen Schritt anzusehen.
Das bedeutet, dass das 3D Modell des Gebäudes in das von Ludwig verwendete Format übersetzt werden muss, welches alle für die Suchopertation notwendigen Informationen über die individuellen Bauteile besitzt.
Somit wird Ludiwgs Format zum "Konvertierten Datenformat", das aus dem 3D Modell berechnet wird.
Das Ergbnis der Suchoperation auf dem Konvertierten Datenformat stellt ein für einen heterogenen Roboterschwarm abarbeitbarer Bauplan dar.
Dafür muss die Suchoperation in Teilen angepasst werden.

Fragestellung: Wie wird aus dem Bauteilplan ein für Roboterschwärme abarbeitbarer Plan zur Erstellung des Gebäudes?

\section{Format zur Beschreibung der Bauteile aus welchen Wände bestehen}
Wie beschreibt man Bauteile, sodass neben der geometrischen Struktur z.B. auch die Berarbeitungsmöglichkeiten (wie etwa das Zerschneiden eines Ziegels) angegeben/eingeschränkt werden können.
Wie verwendet man diese Informationen, um Wände, die aus einem solchen Baustein errichtet werden sollen schon vorab im 3D Editor in die darin vorgegebenen Einschränlungen zu pressen (z.B. Legosteine sollten nicht zerschnitten werden -> es gibt also einen Stein, welcher als kleinste Größenänderung für die Wand gelten muss. Die Wand wird quasi in ein Grid gepresst. Wie sieht das aber aus, wenn es sich nicht um so einfache geometrische Körper handelt? -> sau rechenaufwendig?)