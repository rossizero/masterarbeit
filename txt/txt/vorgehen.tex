\section{Vorgehensweise}


\subsection{Einarbeiten in IFC}
Da sich IFC in der Industrie als Standard durchgesetzt hat, ist dessen Verwendung für diese Arbeit sinnvoll.
Es existieren Beispielprojekte, die zum Testen herangezogen werden können \cite{Examples1:online}.
Durch die Unterstützung von Blender können nach kurzer Einarbeitung in die Projektstruktur von IFC Modellen ebenfalls schnell eigene Projekte umgesetzt werden, die den Anforderungen besser entsprechen.

\subsection{Filtern relevanter Strukturen aus einem IFC Projekt}
Wie bereits erwähnt führt IFC Klassen wie Wände, Türen und Fenster ein. 
Dadurch sollte es möglich sein, das Modell nach Objekten zu durchsuchen, welche mit für diese Arbeit relevante Klassen markiert wurden.
Fenster und Türen stellen eine besondere Herausforderung dar, da ein Roboter, welcher auf einer Wand entlangfahren kann, vermutlich nicht in der Lage ist, diese Lücken in der Wand zu überspringen.
Somit muss das während dem Finden eines Bauplans beachtet werden.

\subsection{Diskretisieren von Strukturen mit vorgegebenen Bauteilklassen}
Es muss eine Lösung gefunden werden, beliebige Wände mit einem vorgegebenen, vermutlich stark limitierten Set an Bauteilklassen abzubilden, ohne das Modell zu verändern.
Da dies nicht immer möglich ist, wird für diese Arbeit in diesem Schritt davon ausgegangen, dass passende Modelle als Input geliefert werden, da das Einschränken des Nutzers im Modellierungsschritt stark von der Realität des Architekturprozesses abweichen würde.
Eine Möglichkeit wäre das "Teildiskretisieren" der Strukturen und ein Ausgabeset an Volumenteilen, für welche keine passende Bauteilklasse gefunden werden konnte (etwa ein Bruchteil eines Ziegels).

\subsection{Einarbeiten in und Evaluierung von Ludiwgs Dissertation}
Da Ludwigs Programmcode sehr wahrscheinlich in Teilen angepasst werden muss, ist eine Evaluierung des aktuellen Stands von Nöten und in welcher Weise dem Nutzer die darin enthaltenen Prozesse zur Verfügung gestellt werden.
Dabei könnte es sich entweder um einen in Blender integrierten Konverter handeln oder um eine Erweiterung des BIM-Servers.
Außerdem gilt es herauszufinden, ob das Anpassen seines Algorithmus an für Roboterschwärme optimierte Baupläne den Rahmen dieser Arbeit sprengen würde und wie sich dessen Output dafür ändern müsste.
Eventuell ist es ausreichend, den Roboterschwarm nicht bei der Suche nach einem geeigneten Bauplan zu berücksichtigen und diesen sogar dahingehend zu optimieren, sondern im Anschluss den Bauplan nach Möglichkeiten zur synchronen Bearbeitung mehrerer Agenten zu untersuchen.

\subsection{Überführung in Ludwigs Format}
Nach der Diskretisierung des Modells in eine Menge an Bauteilen, können diese in das von Ludwigs Dissertation vorgesehene Schema konvertiert werden.
Dazu müssen vor allem die Beziehungen einzelner Bauteile zueinander erkannt und die Constraints der Bauteilklassen und Robotertypen beachtet werden.
Im Anschluss können diese Daten nun verwendet werden, um mithilfe von Ludiwgs Dissertation einen Bauteilplan zu suchen.

\subsection{Format des Ergebnisses festlegen}
Hier müsssen sich Luca und ich über den Verlauf unserer Masterarbeiten immer wieder absprechen und Anforderungen vergleichen.
