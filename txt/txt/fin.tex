\chapter{Fazit und Ausblick}
Mit dem Ergebnis dieser Arbeit konnten erstaunlich viele Teilprobleme bei der Überführung eines 3D Modells eines Gebäudes in konkretes Mauerwerk identifiziert und für viele davon Lösungen gefunden werden.
Vor allem durch die Unterstützung beliebiger Basismodule und das einfache Definieren neuer Mauerwerksverbände unterscheidet sich das erarbeitete Konzept von einigen in Kapitel~\ref{related} vorgestellten Veröffentlichungen.
Mit der flexiblen Definition von Regelsets innerhalb einer Ontologie konnte der Grundstein für eine komplexe regelbasierte Analyse des errechneten Bauplanentwurfs gelegt werden.
Dies konnte anhand von Fallstudien und diverser Szenarien belegt werden.

Dennoch gibt es selbstverständlich einige Erweiterungsmöglichkeiten.
Insbesondere das programmatische Lösen kritischer Bereiche wie etwa T-Kreuzungen und anderen Situationen, die nicht zuvor in den beteiligten Mauerwerksverbänden eingepflegt wurden, stellt ein interessantes aber komplexes Problem dar.
Hier bietet sich das Verwenden von Optimierungsansätzen aus der Domäne des \textit{Bin Packings} oder das darauf gezielte Trainieren von \textit{Machine Learning Algorithmen} an.
Damit könnten eventuell auch Probleme wie abgerundete Ecken und Ecken, die keinen 90\textdegree{} Winkel aufweisen umgesetzt werden.
Der bereits angesprochene Anwendungsfall nicht quadratischer Wandstücke kann mit dem vorliegenden Grundgerüst der schichtweisen Betrachtung einer Wand ebenfalls nachträglich in das Konzept integriert werden.
Die Definition eines Wandabschnitts durch ein arbiträres Vieleck lässt sich leicht durch \glqq{}Ausstanzen\grqq{} dieses Vielecks aus einem quadratischen Wandstück realisieren.
Eine Erweiterung um eine simulative Komponente ist ebenfalls denkbar.
Darin könnten etwa Statik-Berechnungen durchgeführt werden, um potenziell fehlerhafte Bereiche im Gebäudemodell frühzeitig ausfindig zu machen.
Derzeit wird die Fugengröße zwischen Bausteinen durch Abzug eines Bruchteils der Bausteingröße dynamisch errechnet.
Da alle Berechnungen ohnehin auf Basis des angegeben Rastermaßes eines Bausteins getätigt werden, verfälscht dies zwar nicht die errechneten Ergebnisse, aber die exportierten Meshes entsprechen deswegen nicht exakt der Realität.
Es ist möglich diese zusätzliche Information den Bausteindefinitionen hinzuzufügen und bei der Generierung der Meshes zu berücksichtigen.