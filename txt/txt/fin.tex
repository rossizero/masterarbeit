\chapter{Fazit und Ausblick}

\section{Sternchenaufgabe: Szenario mit "runden" Wänden und arbiträren (nicht rechteckige) Bausteinformen / schräge schnitte}
Was machen wir wenn die Wand nicht perfekt mit den vorgegebenen Bausteintypen gebaut werden kann (Beispiel ein runder Turm)
Bausteinverbindungen (wie Mörtel) betrachten und als \textit{Verbindungselement} in das Bausteinformat mit aufnehmen? Wie kann man diesen so einschränken, dass nur physisch machbares ausgrechnet wird.
Was wenn die Bausteine arbiträre Formen haben und nur in sehr komplexen Mustern eine "dichte" Wand ergeben -> tiling Probleme.
