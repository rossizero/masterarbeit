\chapter{Fazit und Ausblick}
Mit dem Ergebnis dieser Arbeit konnten erstaunlich viele Teilprobleme bei der Überführung eines 3D Modells eines Gebäudes in konkretes Mauerwerk identifiziert und für viele davon Lösungen gefunden werden.
Vor allem durch die Unterstützung beliebiger Basismodule und das einfache Definieren neuer Mauerwerksverbände unterscheidet sich das erarbeitete Konzept von einigen in Kapitel~\ref{related} vorgestellten Veröffentlichungen.
Mit der flexiblen Definition von Regelsets innerhalb einer Ontologie konnte der Grundstein für eine komplexe regelbasierte Analyse des errechneten Bauplanentwurfs gelegt werden.
Dies konnte anhand diverser Fallstudien und Szenarien belegt werden.
Dennoch gibt es selbstverständlich einige Erweiterungsmöglichkeiten.
Insbesondere das programmatische Lösen kritischer Bereiche wie etwa T-Kreuzungen und anderen Situationen, die nicht zuvor in den beteiligten Mauerwerksverbänden eingepflegt wurden, stellt ein interessantes aber komplexes Problem dar.
Hier bietet sich das Verwenden von Optimierungsansätzen aus der Domäne des \textit{Bin Packings} oder das darauf gezielte Trainieren von \textit{Machine Learning Algorithmen} an.
Damit könnten eventuell auch Probleme wie abgerundete Ecken und Ecken, die keinen 90\textdegree{} Winkel aufweisen umgesetzt werden.
Aber auch der Anwendungsfall nicht quadratischer Wandstücke kann mit dem vorliegenden Grundgerüst der schichtweisen Betrachtung eines Wandabschnitts durch \glqq{}Ausstanzen\grqq{} eines Vielecks daraus berücksichtigt werden.
Eine Erweiterung um eine simulative Komponente ist ebenfalls denkbar.
Darin könnten etwa Statik-Berechnungen durchgeführt werden, um potenziell fehlerhafte Bereiche im Gebäudemodell ausfindig zu machen.

TODO Rasterlose modelle und Wanddimensionen untersützen, die keinem Raster entsprechen -> das ermöglicht das fremder modelle