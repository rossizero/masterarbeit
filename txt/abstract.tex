Aufgrund des sich zuspitzenden Fachkräftemangels in der Baubranche ist es nicht verwunderlich, dass zunehmend Bestrebungen in Richtung der Automatisierung dieses Industriebereichs angestellt werden.
Möglich sind automatisierte Bauprojekte erst aufgrund der fortschreitenden Digitalisierung und Vereinheitlichung durch internationale Standards.
Denn der Automatisierung liegt oftmals ein nicht zu unterschätzender, vorangehender Planungsaufwand zugrunde, welcher erst dadurch softwareseitig durchführbar wird.
Diese Arbeit beschäftigt sich mit den Herausforderungen ein digitales 3D Modell eines Gebäudes zu erstellen und in einen konkreten Bauplan zu überführen, während trotz angestrebter Nutzerfreundlichkeit größtmöglicher Freiheitsgrad in der Modellierungsphase beibehalten wird.
Dabei liegt der Fokus auf dem Errichten von Mauerwerk mithilfe von Formsteinen.